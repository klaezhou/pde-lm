\documentclass{article}
\usepackage[british]{babel}
\usepackage{amsmath,amssymb,latexsym}

%%%%%%%%%% Start TeXmacs macros
\newcommand{\assign}{:=}
\newcommand{\cdotnorm}{\cdot}
\newcommand{\infixand}{\text{ and }}
\newcommand{\matheuler}{\gamma}
\newcommand{\tmop}[1]{\ensuremath{\operatorname{#1}}}
\newcommand{\tmstrong}[1]{\textbf{#1}}
\newcommand{\tmtextmd}[1]{\text{{\mdseries{#1}}}}
\newcommand{\tmtextrm}[1]{\text{{\rmfamily{#1}}}}
\newcommand{\tmtextup}[1]{\text{{\upshape{#1}}}}
\newenvironment{proof}{\noindent\textbf{Proof\ }}{\hspace*{\fill}$\Box$\medskip}
\newenvironment{tmindent}{\begin{tmparmod}{1.5em}{0pt}{0pt}}{\end{tmparmod}}
\newenvironment{tmparmod}[3]{\begin{list}{}{\setlength{\topsep}{0pt}\setlength{\leftmargin}{#1}\setlength{\rightmargin}{#2}\setlength{\parindent}{#3}\setlength{\listparindent}{\parindent}\setlength{\itemindent}{\parindent}\setlength{\parsep}{\parskip}} \item[]}{\end{list}}
\newenvironment{tmparsep}[1]{\begingroup\setlength{\parskip}{#1}}{\endgroup}
\newtheorem{lemma}{Lemma}
\newtheorem{proposition}{Proposition}
\newtheorem{theorem}{Theorem}

\begin{document}
	
	\section{Stochastic LM Algorithm}
	
	Consider the following least square problem:
	\begin{equation}
		\min_{x \in \mathbb{R}^d} f (x) := \frac{1}{2} \| r (x) \|^2 = \frac{1}{2}
		\sum_{i = 1}^M | r_i (x) |^2
	\end{equation}
	where $r_i$ are continuously differentiable, $i = 1, \dots, M$. Build a set
	$\mathcal{F}_b := \left\{ F_l \subseteq \{1, \dots, M\} \mid |F_l| = b \right\}$, uniformly
	randomly choose $F \in \mathcal{F}_b$.
	
	\begin{equation}
		\min f (x_{F}) := \frac{1}{2} \| r (x_F) \|^2
	\end{equation}
	
	Construct a quadratic function as follows:
	\begin{equation}
		m_k (x_F^k + h) := f (x^k_F) + g^T (x_F^k) h + \frac{1}{2} h^T H
		(x_F^k) h + \frac{1}{2} \mu^k \| h \|^2
	\end{equation}
	where $\mu_k$ is the regularization parameter and $g (x_F^k) = \frac{1}{2}
	\sum_{i = 1}^M \nabla | r_i (x_F) |^2 = \sum_{i = 1}^M \nabla r_i (x_F) r_i
	(x_F)$, $H (x_F^k) = \sum_{i = 1}^M \nabla r_i (x_F) \nabla r_i
	(x_F)^T$. The LM step is obtained by solving the subproblem:
	\[
	\text{argmin}_h m_k (x_F^k + h)
	\]
	i.e., $h_F^k \leftarrow (H (x_F^k) + \mu^k I) h = - g (x_F^k)$. The ratio $\rho$
	is defined as:
	\begin{equation}
		\rho_k = \frac{f (x^k_F) - f (x^k_F + h_F^k)}{m_k (x^k_F) - m_k (x^k_F +
			h_F^k)}
	\end{equation}
	
	\begin{flushleft}
		\tmstrong{Algorithm 1.1\quad Stochastic LM algorithm}\smallskip
		
		\begin{tmindent}
			Step 0: Randomly choose a parameter set $F$ and $\mu > 0$, initialize the
			damping parameter $\mu_0 = \mu \| r (x_F^0) \|^2$, constants
			$\gamma > 1$, $\mu_{\min}$, and $\eta_1, \eta_2 > 0$, set $k = 0$.
			
			\medskip
			
			Step 1: If a stopping criterion is satisfied, go to Step 0 or stop;
			otherwise, go to Step 2.
			
			\medskip
			
			Step 2: Obtain the direction $h_F^k$.
			
			\medskip
			
			Step 3: Compute the ratio $\rho_k$ in (4).
			
			\medskip
			
			Step 4: If $\rho_k \geq \eta_1$ and $\| g (x_F^k) \|^2 \geq \frac{\eta_2}{\mu^k}$,
			set $x^{k + 1}_F \leftarrow x^k_F + h_F^k$ and $\mu^{k + 1} = \max (\mu^k / \gamma, \mu_{\min})$;
			otherwise, set $x_F^{k + 1} = x_F^k$ and $\mu^{k + 1} = \gamma \mu^k$. Then $k = k + 1$.
		\end{tmindent}
	\end{flushleft}
	
	\section{Convergence Analysis}
	
	\begin{proposition}
		Suppose $\tau^k$ is the solution of $\text{argmin}_h m_k (x_F^k + h)$, then
		the following condition holds:
		\begin{equation}
			m_k (x^k_F) - m_k (x^k_F + \tau^k) \geq \frac{1}{4}  \| g (x_F^k)
			\|^2 \min \left\{ \frac{1}{\mu^k}, \frac{1}{\| H (x_F^k) \|} \right\}
		\end{equation}
		and
		\begin{equation}
			\| \tau^k \| \leq \frac{2 \| g (x_F^k) \|}{\mu^k}
		\end{equation}
	\end{proposition}
	
	\begin{proposition}
		Suppose $r_i (x)$ are continuously differentiable and $\nabla r_i (x)$ are
		Lipschitz continuous. If $f (x)$ is bounded and $\| H (x) \| \leq c$ for a
		constant $c > 0$, then there exists a constant Lipschitz coefficient $L > 0$
		and for any $x_F, y_F \in F$, the descent lemma tells us:
		\begin{equation}
			| f (y_F) - f (x_F) - \nabla f (x)^T  (y - x) | \leq \frac{L}{2} \| y
			- x \|^2
		\end{equation}
	\end{proposition}
	
	% Continue with the rest of the document...
